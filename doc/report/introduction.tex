\section{Introduction}
A Vehicular Ad-Hoc Network, or VANET, is a form of Mobile ad-hoc network, to provide communications among nearby vehicles and between vehicles and nearby fixed equipment, usually described as roadside equipment. It will contribute to safer and more efficient roads by providing timely information to drivers and concerned authorities.
As all wireless networks, VANET can be easily subject to any kinds of security  attacks then, for ensure a good security, VANET needs an appropriate security architecture that should be based on different security aspects. One of the most important security aspect on which we focused our work is Authentication and Key management.\\
Our project was to implement a simple frame work for the simulation of secure messages exchange in VANET. The simulator is essentially based on  signing  messages before send them in network area, to be able to verify the integrity of any messages received from the network area. All the messages are sent in broadcast way in a limit area\\
The techniques used to provide these security features are essentially based on two mains approaches provided
 by\cite{calandriello} which are :
\begin{itemize}
\item BaseLine Pseudonyms,\\
In this aspect a considerable amount of certified public key called pseudonym  will stored in a vehicle, in order the ensure the privacy of the sender it will continuously change the public key certificate or pseudonym to sign a given amount of messages or to sign for a limited time this to avoid the sender to be retrieved by some receivers from a subsequent signed messages. These pseudonym should issued by a well known Certificated Authority in order to provide authentication.

\item Hybrib Scheme\\
This technique combines two approaches, coupling the pseudonym generation and the Group signature scheme.
Instead of stored the pseudonyms like in the previous case, the pseudonym  is generated on the fly when it want to send a message, it has to be sign by a group public key provided to any vehicle by some group manager in order to warranty the authentication between user of the same group.
\\
\url{http://www.sigmobile.org/workshops/vanet2007/slides/3.pdf}
\end{itemize}
These two techniques during the implatations has been subject to somes otpimization.
\section{Framework Introduction}
\subsection{Security Implementations}
Befere starting with real implementation of Vanet Simulator we want to focus on security methods. All this technology walk around three main task: authentication, message integrity and no-repudiation; for this reason this technology use asymmetric cryptography. At this time we have two principal techniques: RSA and Elliptic curves. At the begining of this project we have implemented the first technique with 1024 bit of securtity level because is ready to use in java platform. After few steps in this project we reach a lot of problem linked to the nature of this security implementation. In first analysis we have a real good time responses, see table \ref{tab:RSAVelocity} at page \pageref{tab:RSAVelocity}, for sign and verify and authority verification but on the other hand we reach network problems caused by packet dimensions.
\begin{table}[!ht]
	\centering
	\caption{RSA Speed Analysis}
	\begin{tabular}{|c|c|c|c|c|c|c|}
	\hline\hline 
	\textbf{Provider} & \textbf{Sign} & \textbf{Verify} & \textbf{CA verification} & \textbf{sign/s} & \textbf{verify/s}  & \textbf{CA verify/s}\\
	\hline
	Java2 1.6u14 & 0.009284s & 0.000521s & 0.000525s & 107.7 & 1919 & 1904 \\
	\hline
	\hline     %inserts single line 
 	\end{tabular} 
	\label{tab:RSAVelocity}
\end{table}
\section{Certification Authority}
If we consider to send on netowork beacons of 200 bytes with RSA technique we have an enormous crypyography overhead, see table \ref{tab:RSAOverHead} at page \pageref{tab:RSAOverHead}.
\begin{table}[!ht]
	\centering
	\caption{RSA Cryptography Overhead}
	\begin{tabular}{|c|c|c|}
	\hline\hline 
	\textbf{Payload} & \textbf{Signature} & \textbf{Certificate}\\
	\hline
	200 & 128 & 1029 \\
	\hline
	\hline     %inserts single line 
 	\end{tabular} 
	\label{tab:RSAOverHead}
\end{table}
Before passing to Elliptic curves we check a solution on this technology, infact the real problem is the signature length and more of all certificate length. If we want 1024 bit of security level we can not modify signature length but we can working on certificate length, we have constructed a X509 striped certificate for reduce dimension of packet removing all not useful information from certificate. After this step we have reduced dimension to 750 bytes, too big yet. The real problem is length of public key inside certificate and we can not work on that.\\
At this point we have decided to passed on elliptic curves because we are not able to explain motivation of RSA using in this project.
\subsection{Role of Certification Authority and CRL}
Certification authority and CRL for Vanet Simulator is actually a point of discussion\cite{calandriello}. Vanet Simulator implement only one CA and not use certicate revocation list.