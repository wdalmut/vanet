%
% 1. process this file with pdflatex
% 2. remind to process it twice otherwise cross-references will be wrong
%
\documentclass[a4paper,12pt]{article}
%
% This is to create hyperlinks for index, URLs and citations
% (now we can use the command \url{...} to create URL with hyperlink)
% 
\usepackage{color}
\usepackage[a4paper,colorlinks=true,urlcolor=blue,citecolor=blue,linkcolor=blue,bookmarks=false]{hyperref}
%
% This allows inclusion of pictures.
% Create figures with PowerPoint and then export them individually
% in PDF, PNG, JPEG, or GIF format (in order of preference)
%
\usepackage[pdftex]{graphicx}
\DeclareGraphicsExtensions{.pdf,.png,.jpg,.gif}
%
% Definition of margins
%
\usepackage[top=2cm,bottom=2cm,left=2cm,right=2cm]{geometry}
%
% Paragraph skip and indent
%
\setlength\parskip{\medskipamount}
\setlength\parindent{0pt}
%
% Frequently used abbreviations.
% Note that if there is a space after an abbreviation then it must be quoted:
%-  example1: \ie\ this is an example
% - example2: the \ipsec\ protocol
% but tehre is no need for this when there is a punctuation mark
% - example3 (note the full stop at the end): we use SSL rather than \ipsec.
%
\def\eg{e.g.}
\def\ie{i.e.}
\def\ipsec{IPsec}
\def\myfig#1{Fig.~#1}
\def\mytab#1{Tab.~#1}
\def\rfc#1{RFC-#1}% usage: \rfc{1422}
\def\baseline{Baseline Pseudonyms~}
\def\hybrid{Hybrid Scheme~}
\def\vf{Vanet Simulator Framework~}
\def\vs{Vanet Simulator~}
%
\begin{document}

\title{Vanet Simulator
\\
{\normalsize Report for the Computer Security exam at the Politecnico di Torino}
}
\author{Walter Dal Mut (161600)\\Armand Sofack (157938)
\\
{\normalsize tutor: Giorgio Calandriello}
}
\date{June 2009}
\maketitle

\vfill

\rule{\textwidth}{1pt}

\tableofcontents

\rule{\textwidth}{1pt}

\vfill

\newpage
\section{Introduction}
Vehicular networks  (VANET) as all wireless networks can be easily subject to any kinds of security  attacks  then , it need an appropriate security achitecture. One of the principale features is to ensure the integrity of messages exchanged between vehicles.\\
Our project is to implement a simple frame work for the simulation of secure messages exchange in VANET. The simulator will essentially based on  signing  messages  before send its through the network and to able to verify  the integrity of any messages received from the VANET.\\
The techniques used to provide this integrity are essentially based on two mains approachs provided by\cite{calandriello} which are :
\begin{itemize}
\item BaseLine Pseudonyms, a pseudonym is a public certified key,
the basic approach to privacy is based on periodically changing the psuedonym.
\item Hybrib Scheme,is the combination of two approachs coupling the pseudonym generation and the Group signature scheme
\\
\url{http://www.sigmobile.org/workshops/vanet2007/slides/3.pdf}
\end{itemize}
These two techniques during the implatations has been subject to somes otpimization.
\section{UML Diagrams}
\section{UML Diagrams}
\subsection{Framework implementation}
The \vs is based on stack representation, in particoular is composed by a vehicles which a transceiver for send and receive messages on the network and under that leve a security box which use the security implementation which you have set during simulation.
\subsection{Class Diagram}
\begin{figure}[ht]
% If the picture uses fonts of the correct size (10 ... 12 pt)
% then can be included without scaling
%\centerline{\includegraphics{class_diagram.pdf}}
% otherwise see the example in the following (commented out) line
% to scale it relatively to the page width
\centerline{\includegraphics[width=0.9\textwidth]{vanet_class.pdf}}
\caption{Class Diagram}
\label{fig:class_diagram}
\end{figure}
\begin{figure}[ht]
% If the picture uses fonts of the correct size (10 ... 12 pt)
% then can be included without scaling
%\centerline{\includegraphics{class_diagram.pdf}}
% otherwise see the example in the following (commented out) line
% to scale it relatively to the page width
\centerline{\includegraphics[width=0.9\textwidth]{baseline_send_message.pdf}}
\caption{Sequence Diagram \baseline securize and send message}
\label{fig:sequence_send_baseline}
\end{figure}
\subsection{Sequence Diagrams}
\begin{figure}[ht]
% If the picture uses fonts of the correct size (10 ... 12 pt)
% then can be included without scaling
%\centerline{\includegraphics{class_diagram.pdf}}
% otherwise see the example in the following (commented out) line
% to scale it relatively to the page width
\centerline{\includegraphics[width=0.9\textwidth]{baseline_receive_message.pdf}}
\caption{Sequence Diagram \baseline receive and check message}
\label{fig:sequence_receive_baseline}
\end{figure}
\begin{figure}[ht]
% If the picture uses fonts of the correct size (10 ... 12 pt)
% then can be included without scaling
%\centerline{\includegraphics{class_diagram.pdf}}
% otherwise see the example in the following (commented out) line
% to scale it relatively to the page width
\centerline{\includegraphics[width=0.9\textwidth]{hybrid_scheme_send_message.pdf}}
\caption{Sequence Diagram \hybrid securize and send}
\label{fig:sequence_send_hybridscheme}
\end{figure}
\begin{figure}[ht]
% If the picture uses fonts of the correct size (10 ... 12 pt)
% then can be included without scaling
%\centerline{\includegraphics{class_diagram.pdf}}
% otherwise see the example in the following (commented out) line
% to scale it relatively to the page width
\centerline{\includegraphics[width=0.9\textwidth]{hybrid_scheme_receive_message.pdf}}
\caption{Sequence Diagram \hybrid receive and check}
\label{fig:sequence_receive_hybridscheme}
\end{figure}
\section{Security Implementations}
\subsection{\baseline}For the $i$-th pseudonym the CA provide a valid certificate with the public key signed by the CA and a private key for signing messages.\\
The easy way for realize this security implementation is to attach after every message a digital signature and the certificate used for signing message. In this way a vehicle which receive the message can verify the validity of certificate and so verify the digital signature of message. It's really important that pseudonyms is used only for a few time $\tau$ for provide security against the track message reception, because if I use the same certificate for a long period of time an attacker can follow the certitificate and identify the car which send messages. This implementation it's really difficult to implement because the message is too long because if we sum the message length plus the signature plus certificate with public key we have a big packet to send on network. For resolve the overhead problem we can use three optimizations \cite{calandriello}.\\
The main problem of \baseline is based on certificate because this implementation use a preset of certificates, around a thousand, for sign beacons and send on network. For privacy reasons we can not re-use certificates and after a period of time, for that the life-time of \baseline is dependently from certificate preloaded and this is a real problem because we have to insert new certificates after a period of time and for users that could really invasive.
\subsection{\hybrid}
\subsubsection{\hybrid}
Brings the two approaches together the pseudonym generation and the group signature.\\
A Group signature scheme allows each member of the group to anonymously sign a message on behalf of the group. The group members are supposed to be all the vehicles registered with the same Certificate authority. Each vehicle is equipped with a secret group key or also called private group used to sign a group message and a group public that allow a validation of any group signature generated by any group member.  The group keys are installed when the car is manufactures and refreshed at the periodic checkup.\\
 Each node generates its own pseudonym on the fly then uses its private group key to sign the pseudonym in other to generate the corresponding self certificate. This self certificate generated will be verify at the reception by any group member using the group public key.\\ We can also note that in other to ensure the privacy, the group signature allows at each user to not be traced by another group member, but in case of necessity liability is guarantee so, the certificate authority can disclose the identity given a signature transcript.\\
We can also added that, there is no need to reuse the pseudonym because there are generated on the fly and messages can also be linked if needed, but according to some optimization scheme pseudonym could be used for a given amount of time.\\   
The main functions implemented are describe are the following:
\begin{itemize}

\item Securize\\
Basically used to generate a secure payload to sent according the scheme shown in [2] which consist on generate on the fly a pseudonym and try to provide the followings features :
\begin{itemize}
\item Integrity : we generate for each new message  to sent a new pseudonym that will used to sign the message. The signature on that message is added to the payload and also the correspondent self certificate generated by the group signature on the pseudonym generated. 
\item Authentication : The pseudonym  is  certified , signed by the private group key in other to authenticate the sender. But due to the complexity of the group signature algorithm the function to generate the self certify will be dummy taking into account the signature time and the signature bytes size provided by [2]. Both pseudonym  and the signature on it will also added at the message and will be consider as the related certificate.

Finally the packet generated by the function securize will have the following structure:\\
    KeyID is a random number generated to identify the current pseudonym used to sign the payload : 4 bytes.\\
    The payload needed data : 200 bytes \\
    The signature on the payload : 69 bytes\\
    The length of the signature in order to recover it form the final packet : 4 bytes\\
    The self certify of the pseudonym generated on the fly : 298 bytes\\
  Total packet size represented as a vector of bytes is 570 bytes.\\
\item Privacy : the group signature works in other to avoid that some user can be identity through a subsequent messages by it signed.Vehicles are traceable only when using the same pseudonym (as in BaseLine Pseudonym scheme.) 
\end{itemize}

\item Verify\\
The function verify specially check the integrity of the received message and the authenticity of the sender.
Is a Boolean function that receive from the transceiver a message already reconstructed in a class and can extract all additional  information added in the securize  function in order to run the verification .\\
Form these information like the signature performed on the message , the pseudonym  used to signed the message and the self certificate , the verify check  the received message and return true or false in order in the checking is right or wrong.
\end{itemize}

\subsubsection{Optimizations}
We can note in the securize part  that we sent principally two kinds of messages the first one generated in long mode or without optimization and the other one in the short mode according the optimization scheme.The size of the message sent in long mode is 570 respect to the size of the payload, the overhead here is 370.\\
 One of the improvement we did here were to try to reduce that overhead keeping always the security level . The technique implemented  here is base on the  optimization 1\&2 provided by \cite{calandriello}. \\
\begin{itemize}
 \item Optimization 1: At the sender side, the self certificate  is computed only once per pseudonym, because it will remains unchanged throughout the pseudonym lifetime that can be set in a configuration file of the simulator. For the same reason, at the verifier's side the certificate is validated upon the first reception and stored, even though the sender appends it to multiple (all) messages. For all subsequent receptions, if the certificate has already been seen, the verifier skips its validation.
 .\\
 \item Optimization 2: The sender appends  the signature on the message, the pseudonym and the self certificate once  every $\alpha$ messages (beacons); it appends only the signature on the message  on the remaining $\alpha-1$ ones. We call $\alpha$ the Certificate period. In this case the value of $\alpha$ can change through the configuration file. The sort mode message remain  unchanged, and all messages will carry a 4-byte  ID field, that is, a random number indicating which pseudonym must be used to validate the signature on the messsage. The Message ID does not affect privacy as there is a 1:1 correspondence with the pseudonym. When a pseudonym change occurs, the new triplet signature on the message, pseudonym and self certificate  must be computed and transmitted\\
  These two optimization combined allow us to reduce to 72 bytes the overhead for the $\alpha-1$ messages sent after any new computation on the pseudonym, which is relatively significant.
\end{itemize}\section{Simulation Framework implementation}
\section{Simulations and Tests}
Vanet Simulator is written in Java because this programming language is really powerful and flexible to use but more over we see the Java problems with cryptography. First of all the Java security implementation, at this time (version 1.6.x), do not include elliptic curves for asymmetric cryptography and we have research on internet a provider which release elliptic curves and we have found \textit{Bouncy Castle}\footnote{See bibliography at \pageref{bibliography} for more details on this provider}, after this step we have choosen the security level of digital signature and we have opted for 163 bit of \textit{nistb} curves, the less security level in this cryptography provider, after that we have compute with \emph{openssl}\footnote{See bibliography at page \pageref{bibliography} for more details} the tipical speed\footnote{For computation we have used: Pentium 4 Core 2 Duo, processor: Intel T5450 (1.66GHz, 667 Mhz FSB, 2 MB L2 Cache, 2 GB DDR2 RAM} for sign and verify a digital message, see table \ref{tab:OpensslVelocity} at page \pageref{tab:OpensslVelocity}.
\begin{table}[!ht]
	\centering
	\caption{OpenSSL Speed Analysis}
	\begin{tabular}{|c|c|c|c|}
	\hline\hline 
	\textbf{Sign} & \textbf{Verify} & \textbf{sign/s} & \textbf{verify/s} \\
	\hline
	0.0017s & 0.0050s & 583.5 & 199.3 \\
	\hline
	\hline     %inserts single line 
 	\end{tabular} 
	\label{tab:OpensslVelocity}
\end{table}
The Bouncy Castle implementation is slowly than OpenSLL and we have found on time order of difference between C (OpenSSL)  implementation and Java implementation (Bouncy Castle), see table \ref{tab:BouncyCastleVelocity} at page \pageref{tab:BouncyCastleVelocity}.
\begin{table}[!ht]
	\centering
	\caption{Bouncy Castle Speed Analysis}
	\begin{tabular}{|c|c|c|c|}
	\hline\hline 
	\textbf{Sign} & \textbf{Verify} & \textbf{sign/s} & \textbf{verify/s} \\
	\hline
	0.028031s & 0.017342s & 35.6 & 57.6 \\
	\hline
	\hline     %inserts single line 
 	\end{tabular} 
	\label{tab:BouncyCastleVelocity}
\end{table}
In additional analysis we have to verify the certificates with the certification authority and for do that in Java we spent tipically $0.038708s$ roughly $25.83 {verify \over s}$
\subsection{Criptography Overhead}\label{sec:CryptographyOverhead}
In this section we want spent two words around cryptography overhead. During simulation we sent 200 bytes of payload but this payload is securized and for do that we have to insert more bytes for signature and certificate for realize digital signatures. These attachements increase the packet dimensions and for that reason is really useful use optimizations \cite{calandriello}, see table \ref{tab:CryptographyOverhead} at page \pageref{tab:CryptographyOverhead} for more details.
\begin{table}[!ht]
	\centering
	\caption{Cripography Overhead}
	\begin{tabular}{|c|c|}
	\hline\hline 
	\textbf{Signature} & \textbf{Certificate}\\
	\hline
		48 & 234\\
	\hline
	\hline     %inserts single line 
 	\end{tabular} 
	\label{tab:CryptographyOverhead}
\end{table}
\subsection{Real Vanet implementation to Simulator}
Before see the simulation of Vanet, we have to underline the difference between real vanet implementation to simulator. The main task of Vehicular Networks is privacy, integrity and not repudiation of messages and for do that we use a lot of methods, like pseudonyms change or group keys. In the simulator we are obliged to use all network stack from application level to physical level and this feature remove privacy because we are not able to change MAC address dinamically and if we analize the network traffic we can found MAC address and track vehicle moves. Real system do not use the complete network stack and work only on level two (MAC layer) with changing MAC address for every beacon sent on the network.
\subsection{Simulation of Baseline Pseudonyms}
Before starting simulations in baseline psedonyms you have to modify the configuration file \textit{base.properties} \footnote{See the user manual for more information around configurations. Section \ref{usermanual:baseconfiguration} at page \pageref{usermanual:baseconfiguration}} under folder \textit{properties} in root directory of Vanet Simulator.\\
For \baseline we can realize three different optimizations picked up from \cite{calandriello} but in this simulator we have realized the second optimization, in particular the sender append \textit{beacon identification}, \textit{signature}, \textit{public key} and \textit{certificate} only for $\alpha$ messages and transmitt only \textit{beacon identification} and \textit{signature} for remaining $\alpha-1$ beacons. \\
The beacon identification is random number compute on four byte without consider other vehicles, is resonable to use this method whitout remember other IDs is because probability which two vehicle use the same identification number is really quite.
\baseline use optimization two from \cite{calandriello} for limit cryptography overhead, see section \ref{sec:CryptographyOverhead} at page \pageref{sec:CryptographyOverhead} for importance of cryptography overhead, and you can change parameters for realize you personal optimization, for example certificate reattach after tot beacons and certificate or change certificate after a period of time.
\subsection{Performance of Simulator}
For analyze performace of Simulator (figure \ref{fig:performance} at page \pageref{fig:performance}) we have computed statistics for understand the number of signature which we are able to do with the simulator. We have set up the \baseline simulator and set logger into console, after that we have ran system and observed comportaments modifing the number of beacons/sec and the number of vehicles into wireless area.
\begin{figure}[ht]
% If the picture uses fonts of the correct size (10 ... 12 pt)
% then can be included without scaling
%\centerline{\includegraphics{class_diagram.pdf}}
% otherwise see the example in the following (commented out) line
% to scale it relatively to the page width
\centerline{\includegraphics[width=0.5\textwidth]{chart_baseline.pdf}}
\caption{Performace of simulator}
\label{fig:performance}
\end{figure}

\section{Test and profile of simulations}
\section{Comparisons and conclusions}
The work to do was to build a simple as possible a framework that will simulate exchange of secure messages between vehicles in a given area using two principals scheme baseline pseudonym and Hybrid scheme  basically differ  from the way to provide pseudonym  and the related certificate.\\
During the simulation and the development  the behavior of the two scheme were  little different in some points the principal difference has been noted at t he test where for the baseline  pseudonym increasing the sending rate ( number of messages sent per  second by each vehicle) some wrong messages start been appeared due to the impossibility of some vehicle to sign some messages or the overlap time of the life time of two consecutives pseudonyms. The main reason of this can also be due to the fact that simulator run in a same computer because after testing the exchange of the message between two vehicles by installing the framework on two separated computers connected in we have noted the difference and some improvements
This situation is not more present in the hybrid scheme due to the fact that some of the mains functions are not really implemented like the group signature and the group verification that are only dummy function.\\
We can said that up of this basic functionality of the framework, some of the improvements can be added.  In the optimization scheme used in this case is not so efficient because we didn't tacking in account the case where some receiver lost two consecutive messages that can be the last signed by the old pseudonym and the first signed by the  new pseudonym . This could be improved by adding in the combination scheme the optimization 3 provided by [2] and try to implemented in some robust scheme both in sircuze and in verify.\\

\section{Documentation}
\subsection{User Manual}
In this part of this monography we explain the possibilities offred by configurations for using the Vanet Simulator in all of it parts. In particular the main features offered by simulator are changing the security implementation passing by \baseline and \hybrid implementations and configure vehicles on the road, the number of beacons sent during the simulations and modifing logging system for understanding results of simulation.\\
The Vanet Simulator is completely configurable modifying it\'s configurations files under the folder \textit{properties} in the root directory of simulator.
\section{Documentation}
\subsection{User Manual}
In this part of monography we explain the possibilities offered by configurations for using the Vanet Simulator in all of it parts. In particular the main features offered by simulator are changing the security implementation passing by \baseline and \hybrid implementations and configure vehicles on the road, the number of beacons sent during the simulations and modifying logging system for understanding results of simulation.\\
The Vanet Simulator is completely configurable modifying it's configurations files under the folder \textit{properties} in the root directory of simulator.
\subsubsection{Base Configurations}\label{usermanual:baseconfiguration}
The base configurations provide modifications in the core of Vanet Simulator, in particular you can change the \textit{base.properties} file for change core properties like beacons sent, security implementations and others base properties.
If you open the configuration file you see:
\begin{verbatim}
#Max speed in km/h
max_speed = 140
#Max 802.11 cover in meters
wifi_cover = 200
#Access Point broadcast point
server_broadcast_point = 127.255.255.255
#Server Port
server_port = 55055
#Beacons/sec
beacons_sec = 0.1
#If you want no moves of vehicles
no_moves = true
#choose simulator BP or GS
simulator = bp
#max certificate validity into area in seconds
maxCertificateValidityTime = 33
#Reattach certificate every tot beacons
reattachCertificate = 5
#MYSQL properties
mysql_host=127.0.0.1
mysql_username=root
mysql_password=
mysql_database=vanet
#Define the log system
#  0 MYSQL log
#  1 File log
#  2 StdOut log
logSystem=0
\end{verbatim}
For detailed information you can see table \ref{tab:BaseConfiguration} at page \pageref{tab:BaseConfiguration}.
\begin{table}[!ht]
	\centering
	\caption{Base Configuration specifications}
	\begin{tabular}{|c|c|c|}
	\hline\hline 
	\textbf{Property Name} & \textbf{Property Translation} & \textbf{Property Type} \\
	\hline
	max\_speed & The maximum speed of vehicles & int \\
	\hline
	wifi\_cover & The maximum wireless area coverage & int \\
	\hline
	server\_broadcast\_point & The broadcast node for send messages & string \\
	\hline
	server\_port & The port for receive messages & int \\
	\hline
	beacons\_sec & The number of beacons sent in one second & float \\
	\hline
	no\_moves & Lock vehicles into the map & boolean \\
	\hline
	simulator & The simulator which you want use. & string\\
	{} & BP for baseline implementations. & {} \\
	{} & GS for group signature implementation & {} \\
	\hline
	maxCertificateValidityTime & The maximum time for certificate validity & int \\
	\hline
	reattachCertificate & Reattach the certificate every tot beacons & int \\
	\hline
	mysql\_host & The host for mysql & string \\
	\hline
	mysql\_username & Username for authenticate into mysql & string \\
	\hline
	mysql\_password & Password for authenticate into mysql & string \\
	\hline
	mysql\_database & Database to use & string \\
	\hline
	logSystem & The log system which you want use. & int\\
	{} & 0 for mysql log system. & {} \\
	{} & 1 for log data into a files & {} \\
	{} & 2 for log data on console & {} \\
	\hline
	\hline     %inserts single line 
 	\end{tabular} 
	\label{tab:BaseConfiguration}
\end{table}
\subsubsection{Vehicle Configurations}\label{usermanual:vehicleconfigurations}
The vehicles configurations set the status of roads into the simulator. In particular you can modify the number of vehicles into the road, velocity and initial position.\\
The configuration file for vehicle is XML (eXtensible Markup Language) based and is named \textit{vehicles.xml} and positioned into folder \textit{vehicles}; if you open this file you see:
\begin{verbatim}
<?xml version="1.0" encoding="UTF-8"?>
<Vehicles>
    <Vehicle id="1" speed="100" x="10" y="20" />
    <Vehicle id="2" speed="120" x="20" y="50" />
</Vehicles>
\end{verbatim}
Every tag, excluding root tag, identify a new vehicle with attributes like options, in particular you can modify the vehicle identification number changing the \textit{id} attribute or you can change the vehicle speed modifying the \textit{speed} attribute or the position of the vehicle using the \textit{x} or \textit{y} attributes.
For the \baseline~operating mode you have to link certificates and private keys to each vehicles, for doing that you have to follow instruction in section \ref{usermanual:preloadkey} at page \pageref{usermanual:preloadkey}
\subsubsection{Why many log configurations}
The log system for this application is really difficult, in fact the normal std out log system is too slow and produce conflicts if you send a lot of beacons during sign and verify operation but it's really useful because you can understand immediately what the system are doing in real time, the other methods are a middle solution for see result and velocity during sign and verify and the best solution for velocity but difficult to understand in real time the system but it's useful for post-processing. For this reason we have written three type of log system for use the best method when that are compatible with the simulation.
\subsubsection{Log configuration}\label{usermanual:logconfiguration}
The log system use the \textit{log4j} module for write sensible information of simulator. The system provide three log configurations, on standard output stream, file stream or on MySQL database.\\
The configuration of log system it's really powerful and you can set the level of logging or change the log representation for standard out stream or file stream. The configuration of log system is divided into three file, ones for each method and it's collected into folder \textit{properties} which names \textit{stdout.properties} for standard out, \textit{file.properties} for file stream or \textit{mysql.properties} for MySQL database log system.
\subsubsection{MySQL database configuration}
For using MySQL database log system you have to configure the database before launching the Vanet Simulator. You have to create or import a database with tables definition into MySQL using the \textit{vanet.sql} file under the \textit{properties} directory.\\
For create the database and tables definition you have to enter in you MySQL command line and create a new database using command:
\begin{verbatim}
mysql> CREATE DATABASE vanet;
\end{verbatim}
After this step you have to create a table in the new database using commands:
\begin{verbatim}
mysql> use vanet;
...
mysql>CREATE TABLE IF NOT EXISTS `logs` (
  `log_id` int(11) unsigned zerofill NOT NULL auto_increment,
  `level` varchar(255) NOT NULL,
  `class_name` varchar(255) NOT NULL,
  `method_name` varchar(255) NOT NULL,
  `message` text NOT NULL,
  PRIMARY KEY  (`log_id`),
  KEY `level` (`level`)
) ENGINE=InnoDB DEFAULT CHARSET=latin1 AUTO_INCREMENT=1 ;
...
\end{verbatim}
After this step you have configured the MySQL database for record logs from Vanet Simulator.
\subsubsection{Output Reading}
VANET simulator have a log system for showing information, this lines are leveled on five states:
\begin{description}
	\item [trace] Normal trace operation for follow program execution line
	\item [debug] Debug general information
	\item [warn] Warning, little problem, not critical
	\item [error] Severe error
	\item [fatal] Impossibile to run over, the program terminate immediatly
\end{description}
The simulator can provided log  information's in three differents ways, directly on your screen using \textit{std out} or writing into a new file or storing into a database table. All log system show the same information but structured in different ways, a typical std out representation is:
\begin{verbatim}
07:22:22,870 [1632] [Vehicle - 1] (new line for readability reasons)
	DEBUG vanet.Vehicle:105 - (new line for readability reasons)
		Vehicle 1 position, x=37 y=20(new line for readability reasons)
		
07:22:22,870 [1632] [Vehicle - 1] (new line for readability reasons)
	DEBUG vanet.Vehicle:108 - (new line for readability reasons)
		Vehicle 1 send new message(new line for readability reasons)
		
07:22:22,905 [1667] [Vehicle - 2] (new line for readability reasons)
	DEBUG vanet.Vehicle:105 - (new line for readability reasons)
		Vehicle 2 position, x=53 y=50 (new line for readability reasons)
\end{verbatim}
All lines are written with the same logic, in particular start with the time using format: hour, minutes and seconds followed by microseconds. After that into square brackets you have the number of milliseconds elapsed since the start of program, followed by the name of the Java class which execute log, always into square brackets. Then in upper case you have the level of log, followed by the complete name of class followed by double dots and the line of program execution. After that a free text message which inform you on operations, like state changes or errors.\\
Logging into file system follows the same representation and database implementation instead use one table organized in five fields, see table \ref{tab:DBLog} at page \pageref{tab:DBLog}.
\begin{table}[!ht]
	\centering
	\caption{Database table for logging system}
	\begin{tabular}{|c|c|c|c|c|}
	\hline\hline 
	\textbf{Log identification} & \textbf{Level of log} & \textbf{Class name} & \textbf{Method name} & \textbf{Free message}\\
	\hline
	\hline     %inserts single line 
 	\end{tabular} 
	\label{tab:DBLog}
\end{table}
\textbf{Log identification} is an unique number for log entry, other fields represent the same information already expressed for stdout log system, in particular the \textbf{level of log} is always on six step (trace, debug, info, warn, error, fatal), the \textbf{class name} is the name of java class which use the log system, the \textbf{method name} is name of method which call the log system and \textbf{free message} express how the system have logged into database.
\subsubsection{Read vanet simulator operation from log}
Before closing the output reading we want to explain with an example the normal flow of Vanet Simulator with vehicles on the roads. The simulator is multi-thread and don't use syncronization, for that reason the output could be confused. In this section we log pieces of log output but for readability reasons we remove all information which do not have sense for this section, like timestamp and other not useful parts. In addition long line are splitted into two lines, but we consider for one line only those who start with square brackets.\\
\begin{verbatim}
[Vehicle - 1] DEBUG vanet.Vehicle:105 - Vehicle 1 position, x=37 y=20
[Vehicle - 1] DEBUG vanet.Vehicle:108 - Vehicle 1 send new message
[Vehicle - 2] DEBUG vanet.Vehicle:105 - Vehicle 2 position, x=53 y=50
[Vehicle - 2] DEBUG vanet.Vehicle:108 - Vehicle 2 send new message
[Vehicle - 1] DEBUG vanet.security.BaseLinePseudonyms:128 - Message sent in 
  LONG MODE. Certificate expired use new pseudonymous. Certificate ID: 12884
[Vehicle - 2] DEBUG vanet.security.BaseLinePseudonyms:128 - Message sent in 
  LONG MODE. Certificate expired use new pseudonymous. Certificate ID: 54389
...
...
[Vehicle - 2] DEBUG vanet.security.BaseLinePseudonyms:178 - Message sent in 
  SHORT MODE for certificate. Certificate ID: 54389
\end{verbatim}
In the first 6 line, two vehicles have sent new message and these messages are securized using a new certificate (pseudonym), when we use the word \textit{long} we want identify the attach of certificate at the end of message instead with word \textit{short} when we send a message without certificate, in line with optimization 2 in \cite{calandriello}.\\
The system send message in broadcast and for that reason we receive also our messages and in line:
\begin{verbatim}
[Transceiver For Vehicle] INFO  vanet.Transceiver:123 - Skip auto-verification 
  for message: 54389
\end{verbatim}
the transceiver skip verification of self-messages for reduce computational resource. The same operation is executed by the other vehicle. When a vehicle receive a message securized by another vehicle, check it validity and write result of verification on log
\begin{verbatim}
[Transceiver For Vehicle] INFO  vanet.Transceiver:113 - Message secure: 54389
\end{verbatim}
If a signature is not valid or a vehicle do not have certificate for check a message, write a message insecure log:
\begin{verbatim}
[Transceiver For Vehicle] WARN  vanet.Transceiver:118 - Message insecure: 2804
\end{verbatim}
Another information expressed by logs is around positioning of vehicles, in particular when a vehicle go out of wifi range the system express this condition with a message like this:
\begin{verbatim}
[Vehicle - 1] DEBUG vanet.Vehicle:105 - Vehicle 1 position, x=276 y=20
[Vehicle - 1] INFO  vanet.Vehicle:113 - Vehicle 1 is not in the WIFI area, 
  set x position equal to 0
\end{verbatim}
\subsubsection{Install Vanet Simulator on Windows}
For install Vanet simulator on Windows operating system you can use the installer \textit{setupVanetSimulator.exe} and follow the screen information for complete the setup of application.\\
After install procedure you have to open a new console and go into install directory and send the command
\begin{verbatim}
C:\Program File\Vanet Simulator\>java -jar vanetSimulator.jar
\end{verbatim}
After this command you see the bootstrap procedure and after the system run. The default log operation is the standard out and you can see directly all the informations.\\
The common output on screen are this:
\begin{verbatim}
.:: Bootstrap ::.
Loading base properties
Base properties loaded
Loading vehicles configuration
security/certificates/1/c
security/certificates/2/c
Vehicles configuration loaded
.:: Bootstrap end ::.
\end{verbatim}
\subsubsection{Install Vanet Simulator on generic OS}
For install Vanet Simulator you have to setup all folders and executable jar manually. Create new folder in a point of file system and enter in it, after that copy the content of \textit{build} directory and now you can send the command for start the simulator.
\begin{verbatim}
name@domain$ java -jar vanetSimulator.jar
\end{verbatim}
After this command you see the bootstrap procedure and after the system run. The default log operation is the standard out and you can see directly all the informations.\\
The common output on screen are this:
\begin{verbatim}
.:: Bootstrap ::.
Loading base properties
Base properties loaded
Loading vehicles configuration
security/certificates/1/c
security/certificates/2/c
Vehicles configuration loaded
.:: Bootstrap end ::.
\end{verbatim}
\subsubsection{Add certificates and private keys for \baseline}\label{usermanual:preloadkey}
When the system doing the bootstrap in \baseline~mode research certificate and private keys for doing digital signatures.\\
Security properties are in \textit{security} folder, if you add one vehicle you must attach certificates and private keys for work with \baseline~implementation. For do that you have to create a set of folders and positioning certificates and keys in a right place. Under \textit{security} you have one folder named \textit{certificates} and under that folder you have another one folder for each vehicle named with the \textit{vehicle id} (section \ref{usermanual:vehicleconfigurations} at page \pageref{usermanual:vehicleconfigurations}), create new folder named with unique integer identification. Under this folder you have to create another two folders named \textit{c} for certificates and \textit{p} for private keys; after this steps you have to copy your private keys in folder \textit{p} and certificates in folder \textit{c}. Certificates and private keys must have the same name for link, for example: \textit{sec1.crt $\rightarrow$ sec1.key}.tificates and keys in a right place. Under \textit{security} you have one folder named \textit{certificates} and under that folder you have another one folder for each vehicle named with the \textit{vehicle id} (section \ref{usermanual:vehicleconfigurations} at page \pageref{usermanual:vehicleconfigurations}), create new folder named with unique integer identification. Under this folder you have to create another two folders named \textit{c} for certificates and \textit{p} for private keys; after this steps you have to copy your private keys in folder \textit{p} and certificates in folder \textit{c}. Certificates and private keys must have the same name for link, for example: \textit{sec1.crt $\rightarrow$ sec1.key}.


\begin{thebibliography}{99}\label{bibliography}
%
% Citations will be numbered according to the order
% in which they are listed in this section.
%
\bibitem{calandriello} 
P.~Papadimitratos, G.~Calandriello, J.-P.~Hubaux, A.~Lioy,
``Efficient and Robust Pseudonymous Authentication in VANET'',
MOVE 2008: IEEE INFOCOM-2008 workshop on Mobile Networking for Vehicular Environments,
Phoenix (AZ, USA), April 13-18, 2008, pp.~19-27 
\bibitem{raya}
P.~Papadimitratos, M.~Raya, J.-P.~Hubaux,
``Securing Vehicular Communications'',
MOVE 2006: IEEE Wireless Communications Intervehicular Communications,
October, 2006, pp.~8-15 
\bibitem{kargl}F.~Kargl, E.~Schoch, B.~Wiedersheim, T.~Leinmuller,
``Secure and Efficient Beaconing for Vehicular Networks'',
VANET 2008: IEEE INFOCOM-2008 workshop on Mobile Networking for Vehicular Environments,
San Francisco (CA, USA), September 15, 2008, pp.~1-2 
\bibitem{bouncycastle}Bouncy Castle Security Provider,
``http://www.bouncycastle.org'',
Legion of the Bouncy Castle, a provider for the Java Cryptography Extension and the Java Cryptography Architecture
\bibitem{openssl}OpenSSL,
``http://www.openssl.org'',
OpenSSL, C implementation for security.
\end{thebibliography}
\end{document}
%
% Before delivering your report, don't forget to run a spell checker,
% such as aspell (with a UK-english dictionary)
%
