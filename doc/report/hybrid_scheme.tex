\subsubsection{\hybrid}
Brings the two approaches together the pseudonym generation and the group signature.\\
A Group signature scheme allows each member of the group to anonymously sign a message on behalf of the group. The group members are supposed to be all the vehicles registered with the same Certificate authority. Each vehicle is equipped with a secret group key or also called private group used to sign a group message and a group public that allow a validation of any group signature generated by any group member.  The group keys are installed when the car is manufactures and refreshed at the periodic checkup.\\
 Each node generates its own pseudonym on the fly then uses its private group key to sign the pseudonym in other to generate the corresponding self certificate. This self certificate generated will be verify at the reception by any group member using the group public key.\\ We can also note that in other to ensure the privacy, the group signature allows at each user to not be traced by another group member, but in case of necessity liability is guarantee so, the certificate authority can disclose the identity given a signature transcript.\\
We can also added that, there is no need to reuse the pseudonym because there are generated on the fly and messages can also be linked if needed, but according to some optimization scheme pseudonym could be used for a given amount of time.\\   
The main functions implemented are describe are the following:
\begin{itemize}

\item Securize\\
Basically used to generate a secure payload to sent according the scheme shown in [2] which consist on generate on the fly a pseudonym and try to provide the followings features :
\begin{itemize}
\item Integrity : we generate for each new message  to sent a new pseudonym that will used to sign the message. The signature on that message is added to the payload and also the correspondent self certificate generated by the group signature on the pseudonym generated. 
\item Authentication : The pseudonym  is  certified , signed by the private group key in other to authenticate the sender. But due to the complexity of the group signature algorithm the function to generate the self certify will be dummy taking into account the signature time and the signature bytes size provided by [2]. Both pseudonym  and the signature on it will also added at the message and will be consider as the related certificate.

Finally the packet generated by the function securize will have the following structure:\\
    KeyID is a random number generated to identify the current pseudonym used to sign the payload : 4 bytes.\\
    The payload needed data : 200 bytes \\
    The signature on the payload : 69 bytes\\
    The length of the signature in order to recover it form the final packet : 4 bytes\\
    The self certify of the pseudonym generated on the fly : 298 bytes\\
  Total packet size represented as a vector of bytes is 570 bytes.\\
\item Privacy : the group signature works in other to avoid that some user can be identity through a subsequent messages by it signed.Vehicles are traceable only when using the same pseudonym (as in BaseLine Pseudonym scheme.) 
\end{itemize}

\item Verify\\
The function verify specially check the integrity of the received message and the authenticity of the sender.
Is a Boolean function that receive from the transceiver a message already reconstructed in a class and can extract all additional  information added in the securize  function in order to run the verification .\\
Form these information like the signature performed on the message , the pseudonym  used to signed the message and the self certificate , the verify check  the received message and return true or false in order in the checking is right or wrong.
\end{itemize}

\subsubsection{Optimizations}
We can note in the securize part  that we sent principally two kinds of messages the first one generated in long mode or without optimization and the other one in the short mode according the optimization scheme.The size of the message sent in long mode is 570 respect to the size of the payload, the overhead here is 370.\\
 One of the improvement we did here were to try to reduce that overhead keeping always the security level . The technique implemented  here is base on the  optimization 1\&2 provided by \cite{calandriello}. \\
\begin{itemize}
 \item Optimization 1: At the sender side, the self certificate  is computed only once per pseudonym, because it will remains unchanged throughout the pseudonym lifetime that can be set in a configuration file of the simulator. For the same reason, at the verifier's side the certificate is validated upon the first reception and stored, even though the sender appends it to multiple (all) messages. For all subsequent receptions, if the certificate has already been seen, the verifier skips its validation.
 .\\
 \item Optimization 2: The sender appends  the signature on the message, the pseudonym and the self certificate once  every $\alpha$ messages (beacons); it appends only the signature on the message  on the remaining $\alpha-1$ ones. We call $\alpha$ the Certificate period. In this case the value of $\alpha$ can change through the configuration file. The sort mode message remain  unchanged, and all messages will carry a 4-byte  ID field, that is, a random number indicating which pseudonym must be used to validate the signature on the messsage. The Message ID does not affect privacy as there is a 1:1 correspondence with the pseudonym. When a pseudonym change occurs, the new triplet signature on the message, pseudonym and self certificate  must be computed and transmitted\\
  These two optimization combined allow us to reduce to 72 bytes the overhead for the $\alpha-1$ messages sent after any new computation on the pseudonym, which is relatively significant.
\end{itemize}