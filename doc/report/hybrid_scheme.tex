\subsection{\hybrid}
Brings the two approaches together the psuedonym generation and the group signature. Each node generates its own
pseudonym or public key then uses the private group key to sign the pseudounym. The group keys are installed when the car is manufactures and refreshed at the periodic checkup.\\
Each pseudonym is used for a limited amount of time and then discarded.\\
In this scheme there is no need to reuse the pseudonym because there are generated on the fly and messages can also be linked if needed.  
the main functions used in the hybrid scheme are describe are the following:

\begin{itemize}

\item Securize\\
Basically used to generate a secure payload to sent according the scheme shown in [2] which consist on generate on the fly a pseudonym and try to provide the followings features :
\begin{itemize}
\item Integrity : we generate for each new message  to sent a new pseudonym that will used to sign the message. The signature on that message is added to the payload and also the corrispondent self certifcate generated by the group signature on the pseudonym generated. 
\item Authentication : The pseudonym  is  certified , signed by the private group key in other to authenticate the sender. But due to the complexity of the group signature algorithm the function to generate the self certify will be dummy taking into account the signature time and the signature bytes size provided by [2]. Both pseudounym  and the signature on it will also added at the message and will be consider as the related certificate.

Finally the packet generated by the function securize will have the following structure:\\
    Packet id : 4 bytes in order to identify the signed packet.\\
    The payload needed data : 200 bytes \\
    The signature on the payload : 69 bytes\\
    The length of the signature in order to recover it form the final packet : 4 bytes\\
    The self certify of the pseudonym generated on the fly : 298 bytes\\
  Total packet size represented as a vector of bytes is 570 bytes.\\
\item Privacy : the group signature works in other to avoid that some user can be identity through a subsequent messages by it signed.Vehicles are traceable only when using the samepseudonym (as in BaseLine Pseudonym scheme.) 
\end{itemize}

\item Verify\\
The function verify specially check the integrity of the received message and the authenticity of the sender.
Is a Boolean function that receive from the transceiver a message already reconstructed in a class and can extract all additional  information added in the securize  function in order to run the verification .\\
Form these information like the signature performed on the message , the pseudonym  used to signed the message and the self certificate , the verify check  the received message and return true or false in order in the checking is right or wrong.

\item Optimization\\

We can note in the securize part  that we sent principaly two kinds of messages the first one generated in long mode or without optimization and the other one in the short modeaccording the optimization scheme.The size of the message sent in long mode is 570 respect to the size of the payload, the overhead here is 370.\\
 One of the improvement we did here were to try to reduce that overhead keeping always the security level . The technique implemented  here is base on the  optimization 1\&2 provided by [2]. \\
 \begin{itemize}
 \item Optimization 1: At the sender side, the self certificate  is computed only once per pseudonym, because it will remains unchanged throughout the pseudonym lifetime that can be se in a configuration file of the simulator. For the same reason, at the verifier's side the certificate is validated upon the first reception and stored, even though the sender appends it to multiple (all) messages. For all subsequent receptions, if the certificate has already been seen, the verifier skips its validation.
 .\\
    We have defined also a  counter for the number of messages sent using the same pseudonym, this allow us to don't need to attach the self certify for a given  number of messages set by the counter and will reduce for about 298 bytes the size of the message sent.\\
   At the verification we choose to verify first the size of the received message  that allow us to check if or not the certificate where attached at the sender  and we decide or not  to skip the authentication.\\ 
The counter value and the timer value can be set  in the configuration file.

\end{itemize}


